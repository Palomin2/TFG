\documentclass[12pt]{article}
\usepackage[letterpaper,top=2cm,bottom=2cm,left=3cm,right=3cm,marginparwidth=1.75cm]{geometry}
\usepackage[spanish]{babel}
\usepackage[utf8]{inputenc}
\usepackage[T1]{fontenc}
\usepackage{fancyhdr}
\usepackage{bm}
\usepackage{mathtools}
\usepackage{graphicx}
\usepackage{bbold}
\usepackage{yhmath} % pour pouvoir obtenir le symbole des ouverts avec une parenthèse
\usepackage{amsmath}    % les symboles les plus fréquents
\usepackage{amssymb}    % des symboles
%deux packages qui étaient dans le truc de Nou
\usepackage{enumitem} % pour les listes
\usepackage{hyperref} %hyperliens

\usepackage{theorem}
\newtheorem{theorem}{Teorema}
\newtheorem{lemma}{Lema}
\newtheorem{observation}{Observaci\'on}
\newtheorem{proof}{Demostraci\'on}
\newtheorem{proposition}{Proposici\'on}
\newtheorem{definition}{Definici\'on}
\newtheorem{corollary}{Corolario}
\DeclareMathOperator\supp{supp } %pour faire des jolis supports

%il suffira de renommer 3 commandes par celles utilisées pour th, lemme, preuve si on veut changer le style des démonstrations notamment. package amsth + \theoremstyle ?


%environnement pour les démo sans compteur et en écriture normale :
\renewenvironment{proof}{%

\vspace{0.5cm}

  \noindent \textbf{Demostraci\'on :}
}{%

  ~\hfill{$\Box$}
}




\newcommand{\E}{\mathrm{E}}
\newcommand{\N}{\mathbb{N}}
\newcommand{\C}{\mathbb{C}}
\newcommand{\Var}{\mathrm{Var}}
\newcommand{\subg}{\mathrm{SubG}}
\newcommand{\R}{\mathbb{R}}
\newcommand{\pp}{\mathbb{P}}

\begin{document}

\section{TABLAS}
\subsection{p=2}

\begin{center}
\begin{tabular} {|c|c|c|c|c|}
\hline
$h$ & Error Norma L2 & Error Norma infinito & Error Norma H1 & Tiempo de ejecución(ns)\\
\hline
$10$ &   $1.0984e-04$ & $1.9700e-04$ & $7.7051e-03$  & $1150300$ \\
$20$ &   $1.5722e-05$ &  $3.1000e-05$ & $1.6852e-03$ & $1979700$   \\
$40$ &   $1.9573e-06$ &  $4.0000e-06$ & $4.1137e-04$ & $3639500$   \\
$80$ &   $4.5354e-07$ &  $1.0000e-06$ & $1.4515e-04$ & $5977700$   \\

\hline
\end{tabular}
\vspace{3cm}
\end{center}


\subsection{p=4}
\begin{center}
\begin{tabular} {|c|c|c|c|c|}
\hline
$h$ & Error Norma L2 & Error Norma infinito & Error Norma H1 & Tiempo de ejecución(ns)\\
\hline
$10$ &   $5.9898e-07$ & $1.0000e-06$  & $2.0412e-05$ & $1409000$ \\
$20$ &   $0$   &  $0$ & $5.1158e-06$ & $3111600$   \\
$40$ &   $0$   &  $0$ & $2.7355e-07$ & $4555700$   \\
$80$ &   $0$   &  $0$ & $3.8227e-15 $ & $9696100$   \\

\hline
\end{tabular}
\vspace{3cm}
\end{center}

\section{CAMBIO DE VARIABLE 1D}
\subsection{Formulación clasica}

Espacio parametrico:

\begin{equation*}
-u''(x) = f(x), \quad x \in \Omega .
\end{equation*}\\

Espacio fisico:

\begin{equation*}
-\frac{1}{J(\xi)} \frac{d}{d\xi} \left( \frac{1}{J(\xi)} u'(r(\xi)) \right) = f(r(\xi)), \quad \xi \in \Omega^{\xi} .
\end{equation*}\\

Equivalentemente:

\begin{equation*}
-\frac{d}{d\xi} \left( \frac{1}{J(\xi)} u'(r(\xi)) \right) = f(r(\xi))J(\xi), \quad \xi \in \Omega^{\xi} .
\end{equation*}\\


\subsection{Formulación variacional}
suponemos $\Omega$ = (0,L), spg\\
Espacio parametrico:

\begin{equation*}
\int_{\Omega} u'(x)v'(x)dx-u'(L)v(L)+u'(0)v(0) = \int_{\Omega} f(x)v(x)dx, \quad x \in \Omega .
\end{equation*}\\

Espacio fisico:

\begin{align*}
&\int_{\Omega^{\xi}} \frac{1}{J(\xi)} u'(r(\xi)) \frac{1}{J(\xi)} v'(r(\xi)) J(\xi) d\xi x-\frac{1}{J(L)}u'(r(L))v(r(L))+\frac{1}{J(0)}u'(r(0))v(r(0)) =\\
&=\int_{\Omega^{\xi}} f(r(\xi))v(r(\xi)) J(\xi) d\xi, \quad \xi \in \Omega^{\xi} .
\end{align*}\\

Simplificando obtenemos:
\begin{align*}
&\int_{\Omega^{\xi}} \frac{1}{J(\xi)} u'(r(\xi))  v'(r(\xi)) d\xi x-\frac{1}{J(L)}u'(r(L))v(r(L))+\frac{1}{J(0)}u'(r(0))v(r(0)) =\\
&=\int_{\Omega^{\xi}} f(r(\xi))v(r(\xi)) J(\xi) d\xi, \quad \xi \in \Omega^{\xi} .
\end{align*}\\

Donde $J(\xi)$ = $\|r'(\xi)\|_{2}$.

\end{document}

