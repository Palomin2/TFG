\documentclass[12pt]{article}
\usepackage{graphicx}
\usepackage{svg}
\usepackage{subcaption}
\usepackage{float}
\usepackage[letterpaper,top=2cm,bottom=2cm,left=3cm,right=3cm,marginparwidth=1.75cm]{geometry}
\usepackage[spanish]{babel}
\usepackage[utf8]{inputenc}
\usepackage[T1]{fontenc}
\usepackage{fancyhdr}
\usepackage{bm}
\usepackage{mathtools}
\usepackage{bbold}
\usepackage{yhmath} % pour pouvoir obtenir le symbole des ouverts avec une parenthèse
\usepackage{amsmath}    % les symboles les plus fréquents
\usepackage{amssymb}    % des symboles
%deux packages qui étaient dans le truc de Nou
\usepackage{enumitem} % pour les listes
\usepackage{hyperref} %hyperliens

\usepackage{theorem}
\newtheorem{theorem}{Teorema}
\newtheorem{lemma}{Lema}
\newtheorem{observation}{Observaci\'on}
\newtheorem{proof}{Demostraci\'on}
\newtheorem{proposition}{Proposici\'on}
\newtheorem{definition}{Definici\'on}
\newtheorem{corollary}{Corolario}
\DeclareMathOperator\supp{supp } %pour faire des jolis supports

%il suffira de renommer 3 commandes par celles utilisées pour th, lemme, preuve si on veut changer le style des démonstrations notamment. package amsth + \theoremstyle ?


%environnement pour les démo sans compteur et en écriture normale :
\renewenvironment{proof}{%

\vspace{0.5cm}

  \noindent \textbf{Demostraci\'on :}
}{%

  ~\hfill{$\Box$}
}




\newcommand{\E}{\mathrm{E}}
\newcommand{\N}{\mathbb{N}}
\newcommand{\C}{\mathbb{C}}
\newcommand{\Var}{\mathrm{Var}}
\newcommand{\subg}{\mathrm{SubG}}
\newcommand{\R}{\mathbb{R}}
\newcommand{\pp}{\mathbb{P}}

\begin{document}
\section{Problema 2D}
\subsection{Formulación clásica y variacional}
\subsubsection*{Espacio físico}
\addcontentsline{toc}{subsubsection}{Espacio físico}
\medskip
\textbf{Formulación clásica:\\}
    Sea $\Omega \subset \mathbb{R}^2$ la geometría definida por NURBS. 
Buscamos $u : \Omega \to \mathbb{R}$ tal que
\[
\begin{cases}
- \Delta u = f & \text{en } \Omega, \\[6pt]
u = g & \text{en } \Gamma_D, \quad \text{(condición de Dirichlet)}, \\[6pt]
\nabla u \cdot n = h & \text{en } \Gamma_N \quad \text{(condición de Neumann)}, \\[6pt]
\alpha u + \beta \, \nabla u \cdot n = r & \text{en } \Gamma_R 
\quad \text{(condición de Robin)}.
\end{cases}
\]


\medskip

\textbf{Formulación variacional (homogeneizada):}

Definimos los espacios
\[
V = \{ v \in H^1(\Omega) \}, 
\qquad 
V_0 = \{ v \in H^1(\Omega) : v=0 \ \text{en } \Gamma_D \}.
\]

Buscamos $u \in V$ con $u=g$ en $\Gamma_D$ tal que
\[
a(u,v) = \ell(v) \qquad \forall v \in V_0,
\]
donde
\[
a(u,v) = \int_\Omega \nabla u \cdot \nabla v \, d\Omega
          + \int_{\Gamma_R} \tfrac{\alpha}{\beta}\, u v \, d\Gamma,
\]
\[
\ell(v) = \int_\Omega f \, v \, d\Omega 
          + \int_{\Gamma_N} h \, v \, d\Gamma
          + \int_{\Gamma_R} \tfrac{r}{\beta}\, v \, d\Gamma.
\]

\subsection{Transformación del espacio físico al paramétrico}

La geometría se describe mediante un mapeo NURBS
\[
\mathbf{x}(\xi,\eta) = \sum_{i=1}^{n} R_i(\xi,\eta)\, \mathbf{P}_i,
\]
donde $R_i(\xi,\eta)$ son las funciones base NURBS y $\mathbf{P}_i$ los 
puntos de control.

\medskip

El cambio de variable entre $\Omega$ y $\hat{\Omega}$ se realiza a través del 
jacobiano de la transformación
\[
J(\xi,\eta) = \frac{\partial \mathbf{x}}{\partial (\xi,\eta)} 
= 
\begin{bmatrix}
\dfrac{\partial x}{\partial \xi} & \dfrac{\partial x}{\partial \eta} \\[8pt]
\dfrac{\partial y}{\partial \xi} & \dfrac{\partial y}{\partial \eta}
\end{bmatrix}.
\]

\medskip

Así, una integral en el espacio físico se transforma en el espacio paramétrico como
\[
\int_{\Omega} f(\mathbf{x}) \, d\Omega 
= \int_{\hat{\Omega}} f(\mathbf{x}(\xi,\eta)) \, \big|\det J(\xi,\eta)\big| \, d\xi \, d\eta.
\]

\medskip

Del mismo modo, los gradientes de las funciones base en coordenadas físicas se obtienen mediante
\[
\nabla_{\mathbf{x}} R_i = J^{-T} \, \nabla_{(\xi,\eta)} R_i,
\]
donde 
\(
\nabla_{(\xi,\eta)} R_i =
\begin{bmatrix}
\dfrac{\partial R_i}{\partial \xi} \\[6pt]
\dfrac{\partial R_i}{\partial \eta}
\end{bmatrix}.
\)

\medskip

En consecuencia, la formulación variacional en el espacio físico
\[
a(u,v) = \int_{\hat{\Omega}} (\nabla_{(\xi,\eta)} u)^T \, G(\xi,\eta) \, (\nabla_{(\xi,\eta)} v) \; |\det J(\xi,\eta)| \, d\xi d\eta
          + \int_{\hat{\Gamma}_R} \frac{\alpha}{\beta}\, u v \; \|J_\Gamma\| \, d\hat s,
\]
\[
\ell(v) = \int_{\hat{\Omega}} f(\mathbf{x}(\xi,\eta))\, v \; |\det J(\xi,\eta)| \, d\xi d\eta
          + \int_{\hat{\Gamma}_N} h \, v \; \|J_\Gamma\| \, d\hat s
          + \int_{\hat{\Gamma}_R} \frac{r}{\beta}\, v \; \|J_\Gamma\| \, d\hat s.
\]

Aquí \(\hat{\Gamma}_N, \hat{\Gamma}_R\) son las fronteras correspondientes en el espacio paramétrico y
\(\|J_\Gamma\|\) es el jacobiano tangente que convierte la integral de línea de \(\hat{\Gamma}\) a \(\Gamma\) en el espacio físico.

\end{document}