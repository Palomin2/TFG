\documentclass[12pt]{article}
\usepackage{graphicx}
\usepackage{svg}
\usepackage{subcaption}
\usepackage{float}
\usepackage[letterpaper,top=2cm,bottom=2cm,left=3cm,right=3cm,marginparwidth=1.75cm]{geometry}
\usepackage[spanish]{babel}
\usepackage[utf8]{inputenc}
\usepackage[T1]{fontenc}
\usepackage{fancyhdr}
\usepackage{bm}
\usepackage{mathtools}
\usepackage{bbold}
\usepackage{yhmath} % pour pouvoir obtenir le symbole des ouverts avec une parenthèse
\usepackage{amsmath}    % les symboles les plus fréquents
\usepackage{amssymb}    % des symboles
%deux packages qui étaient dans le truc de Nou
\usepackage{enumitem} % pour les listes
\usepackage{hyperref} %hyperliens

\usepackage{theorem}
\newtheorem{theorem}{Teorema}
\newtheorem{lemma}{Lema}
\newtheorem{observation}{Observaci\'on}
\newtheorem{proof}{Demostraci\'on}
\newtheorem{proposition}{Proposici\'on}
\newtheorem{definition}{Definici\'on}
\newtheorem{corollary}{Corolario}
\DeclareMathOperator\supp{supp } %pour faire des jolis supports

%il suffira de renommer 3 commandes par celles utilisées pour th, lemme, preuve si on veut changer le style des démonstrations notamment. package amsth + \theoremstyle ?


%environnement pour les démo sans compteur et en écriture normale :
\renewenvironment{proof}{%

\vspace{0.5cm}

  \noindent \textbf{Demostraci\'on :}
}{%

  ~\hfill{$\Box$}
}




\newcommand{\E}{\mathrm{E}}
\newcommand{\N}{\mathbb{N}}
\newcommand{\C}{\mathbb{C}}
\newcommand{\Var}{\mathrm{Var}}
\newcommand{\subg}{\mathrm{SubG}}
\newcommand{\R}{\mathbb{R}}
\newcommand{\pp}{\mathbb{P}}

\begin{document}



\section{CAMBIO DE VARIABLE 1D}
\subsection{Formulación clasica}

Espacio parametrico:

\begin{equation*}
-u''(x) = f(x), \quad x \in \Omega .
\end{equation*}\\

Espacio fisico:

\begin{equation*}
-\frac{1}{J(\xi)} \frac{d}{d\xi} \left( \frac{1}{J(\xi)} u'(r(\xi)) \right) = f(r(\xi)), \quad \xi \in \Omega^{\xi} .
\end{equation*}\\

Equivalentemente:

\begin{equation}
-\frac{d}{d\xi} \left( \frac{1}{J(\xi)} u'(r(\xi)) \right) = f(r(\xi))J(\xi), \quad \xi \in \Omega^{\xi} .
\end{equation}\\


\subsection{Formulación variacional}
suponemos $\Omega$ = (0,L), spg\\
Espacio parametrico:

\begin{equation*}
\int_{\Omega} u'(x)v'(x)dx-u'(L)v(L)+u'(0)v(0) = \int_{\Omega} f(x)v(x)dx, \quad x \in \Omega .
\end{equation*}\\

Espacio fisico:

\begin{align*}
&\int_{\Omega^{\xi}} \frac{1}{J(\xi)} u'(r(\xi)) \frac{1}{J(\xi)} v'(r(\xi)) J(\xi) d\xi x-\frac{1}{J(L)}u'(r(L))v(r(L))+\frac{1}{J(0)}u'(r(0))v(r(0)) =\\
&=\int_{\Omega^{\xi}} f(r(\xi))v(r(\xi)) J(\xi) d\xi, \quad \xi \in \Omega^{\xi} .
\end{align*}\\

Simplificando obtenemos:
\begin{align*}
&\int_{\Omega^{\xi}} \frac{1}{J(\xi)} u'(r(\xi))  v'(r(\xi)) d\xi x-\frac{1}{J(L)}u'(r(L))v(r(L))+\frac{1}{J(0)}u'(r(0))v(r(0)) =\\
&=\int_{\Omega^{\xi}} f(r(\xi))v(r(\xi)) J(\xi) d\xi, \quad \xi \in \Omega^{\xi} .
\end{align*}\\

Donde $J(\xi)$ = $\|r'(\xi)\|_{2}$.

\section{Evaluaciones y resultados de convergencia}

Tomaremos $f(x)=k m^2\pi^2sin(m \pi x), \quad x \in [0,1].$\\
Donde $k=100, m=10$.\\

\subsection{Grado polinomial p=2:}
Tomaremos los puntos de control: ${(0, 0.75, 1)}$\\
Y su respectivo vector de pesos: $ (1,1,1)$\\
Con vector de nodos: ${(0,0,0,1,1,1)}$\\
\subsubsection{Numero de elementos h=10.}
\begin{figure}[H]
    \centering
    \begin{subfigure}{0.32\textwidth}
        \includesvg[width=\linewidth]{ej1_p2_h10_sol.svg}
    \end{subfigure}
    \hfill
    \begin{subfigure}{0.32\textwidth}
        \includesvg[width=\linewidth]{ej1_p2_h10_ders.svg}
    \end{subfigure}
    \hfill
    \begin{subfigure}{0.32\textwidth}
        \includesvg[width=\linewidth]{ej1_p2_h10_curve.svg}
    \end{subfigure}
    \caption{Comparación de resultados.}
    \label{fig:tresgraficas}
\end{figure}
Resultados numéricos concretos:\\
Tiempo de ejecución: 780800 nanosegundos\\
Error Norma L2 = 32.250\\
Error Norma H1 = 1249.3\\
Error Norma Infinito = 140.30\\
\subsubsection{Numero de elementos h=20.}
\begin{figure}[H]
    \centering
    \begin{subfigure}{0.32\textwidth}
        \includesvg[width=\linewidth]{ej1_p2_h20_sol.svg}
    \end{subfigure}
    \hfill
    \begin{subfigure}{0.32\textwidth}
        \includesvg[width=\linewidth]{ej1_p2_h20_ders.svg}
    \end{subfigure}
    \hfill
    \begin{subfigure}{0.32\textwidth}
        \includesvg[width=\linewidth]{ej1_p2_h20_curve.svg}
    \end{subfigure}
    \caption{Comparación de resultados.}
    \label{fig:tresgraficas}
\end{figure}
Resultados numéricos concretos:\\
Tiempo de ejecución: 2267400 nanosegundos\\
Error Norma L2 = 4.3739\\
Error Norma H1 = 344.65\\
Error Norma Infinito = 22.692\\
\subsubsection{Numero de elementos h=40.}
\begin{figure}[H]
    \centering
    \begin{subfigure}{0.32\textwidth}
        \includesvg[width=\linewidth]{ej1_p2_h40_sol.svg}
    \end{subfigure}
    \hfill
    \begin{subfigure}{0.32\textwidth}
        \includesvg[width=\linewidth]{ej1_p2_h40_ders.svg}
    \end{subfigure}
    \hfill
    \begin{subfigure}{0.32\textwidth}
        \includesvg[width=\linewidth]{ej1_p2_h40_curve.svg}
    \end{subfigure}
    \caption{Comparación de resultados.}
    \label{fig:tresgraficas}
\end{figure}
Resultados numéricos concretos:\\
Tiempo de ejecución: 3882400 nanosegundos\\
Error Norma L2 = 0.2922\\
Error Norma H1 = 63.083\\
Error Norma Infinito = 1.4318\\
\subsubsection{Numero de elementos h=80.}
\begin{figure}[H]
    \centering
    \begin{subfigure}{0.32\textwidth}
        \includesvg[width=\linewidth]{ej1_p2_h80_sol.svg}
    \end{subfigure}
    \hfill
    \begin{subfigure}{0.32\textwidth}
        \includesvg[width=\linewidth]{ej1_p2_h80_ders.svg}
    \end{subfigure}
    \hfill
    \begin{subfigure}{0.32\textwidth}
        \includesvg[width=\linewidth]{ej1_p2_h80_curve.svg}
    \end{subfigure}
    \caption{Comparación de resultados.}
    \label{fig:tresgraficas}
\end{figure}
Resultados numéricos concretos:\\
Tiempo de ejecución: 4768000 nanosegundos\\
Error Norma L2 = 0.031083\\
Error Norma H1 = 13.567\\
Error Norma Infinito = 0.1659\\
\subsection{Grado polinomial p=4:}
Tomaremos los puntos de control: ${(0, 0.05, 0.10, 0.95, 1)}$\\
Y su respectivo vector de pesos: $ (1,1,1,1,1 )$\\
Con vector de nodos: ${(0,0,0,0,0,1,1,1,1,1)}$\\
\subsubsection{Numero de elementos h=10.}
\begin{figure}[H]
    \centering
    \begin{subfigure}{0.32\textwidth}
        \includesvg[width=\linewidth]{ej2_p4_h10_sol.svg}
    \end{subfigure}
    \hfill
    \begin{subfigure}{0.32\textwidth}
        \includesvg[width=\linewidth]{ej2_p4_h10_ders.svg}
    \end{subfigure}
    \hfill
    \begin{subfigure}{0.32\textwidth}
        \includesvg[width=\linewidth]{ej2_p4_h10_curve.svg}
    \end{subfigure}
    \caption{Comparación de resultados.}
    \label{fig:tresgraficas}
\end{figure}
Resultados numéricos concretos:\\
Tiempo de ejecución: 1370100 nanosegundos\\
Error Norma L2 = 35.806\\
Error Norma H1 = 1599.8\\
Error Norma Infinito = 124.20\\
\subsubsection{Numero de elementos h=20.}
\begin{figure}[H]
    \centering
    \begin{subfigure}{0.32\textwidth}
        \includesvg[width=\linewidth]{ej2_p4_h20_sol.svg}
    \end{subfigure}
    \hfill
    \begin{subfigure}{0.32\textwidth}
        \includesvg[width=\linewidth]{ej2_p4_h20_ders.svg}
    \end{subfigure}
    \hfill
    \begin{subfigure}{0.32\textwidth}
        \includesvg[width=\linewidth]{ej2_p4_h20_curve.svg}
    \end{subfigure}
    \caption{Comparación de resultados.}
    \label{fig:tresgraficas}
\end{figure}
Resultados numéricos concretos:\\
Tiempo de ejecución: 4262500 nanosegundos\\
Error Norma L2 = 4.2303\\
Error Norma H1 = 293.33\\
Error Norma Infinito = 19.396\\
\subsubsection{Numero de elementos h=40.}
\begin{figure}[H]
    \centering
    \begin{subfigure}{0.32\textwidth}
        \includesvg[width=\linewidth]{ej2_p4_h40_sol.svg}
    \end{subfigure}
    \hfill
    \begin{subfigure}{0.32\textwidth}
        \includesvg[width=\linewidth]{ej2_p4_h40_ders.svg}
    \end{subfigure}
    \hfill
    \begin{subfigure}{0.32\textwidth}
        \includesvg[width=\linewidth]{ej2_p4_h40_curve.svg}
    \end{subfigure}
    \caption{Comparación de resultados.}
    \label{fig:tresgraficas}
\end{figure}
Resultados numéricos concretos:\\
Tiempo de ejecución: 6218200 nanosegundos\\
Error Norma L2 = 0.028325\\
Error Norma H1 = 5.6622\\
Error Norma Infinito = 0.1146\\
\subsubsection{Numero de elementos h=80.}
\begin{figure}[H]
    \centering
    \begin{subfigure}{0.32\textwidth}
        \includesvg[width=\linewidth]{ej2_p4_h80_sol.svg}
    \end{subfigure}
    \hfill
    \begin{subfigure}{0.32\textwidth}
        \includesvg[width=\linewidth]{ej2_p4_h80_ders.svg}
    \end{subfigure}
    \hfill
    \begin{subfigure}{0.32\textwidth}
        \includesvg[width=\linewidth]{ej2_p4_h80_curve.svg}
    \end{subfigure}
    \caption{Comparación de resultados.}
    \label{fig:tresgraficas}
\end{figure}
Resultados numéricos concretos:\\
Tiempo de ejecución: 11270600 nanosegundos\\
Error Norma L2 = 5.6460e-04\\
Error Norma H1 = 0.2017\\
Error Norma Infinito = 2.6000e-03\\

\subsubsection{Información importante sobre convergencias}
Todos los datos referentes a la integración de normas han sido realizados usando una cuadratura de Gauss-Lengendre de 200 elementos,
los elementos de las submatrices del metodo de IGA han sido calculados mediante el método de cuadratura de Gauss-Legendre con p+2 elementos.
\section{Curva Genérica en R3 y solución analitica}
\subsection{Planteamiento y busqueda de solución analitica}
De la sección anterior podemos observar como al definir f sobre el espacio paramétrico esta se "deforma", al igual que hace la recta,
es decir, sufre una reparametrización equivalente a la que sufre el sgmento de recta parametrizado sobre el arco en $[0,1]$, adicionalmente,
es importante notar que si tomamos el segmento de recta entre $[0,b]$ en el espacio fisico como un reescalado de la recta fisica que teniamos previamente
definida en $[0,1]$ bajo las mismas condiciones y sobre el mismo espacio parametrico, notamos que para cualquier punto de $\xi$ que parametriza nuestro segmento de recta,
el jacobiano en este punto es reescalado por b, luego nuestra solución analítica será, en este caso, teniendo en cuenta la ecuación (1) e integrando a cada lado y equiparando
el jacobiano original multiplicado por b con nuestro jacobiano reescalado, implica que nuestra solución es la misma para cada valor del parametro fisico multiplicada por $b^2$ y la de
la derivada por b (añadire ecuaciones sobre esto para aclararlo)\\ 
luego nos da pie a definir sobre el mismo problema en el espacio parametrico pero con un espacio fisico dado por una recta en el espacio de longitud L la cual reparametrizaremos por el arco,
teniendo así que en nuestra reparametrización obtenemos que la solución analítica en cada valor de esta reparametrización es el de nuestra solución sobre la recta de R1 multiplicada por $L^2$.
\subsection{Evaluación del correcto funcionamiento en una curva genérica}
Antes de iniciar aclarar que debido a no tener el resultado analítico exacto de la longitud de la curva no podemos calcular aquí resultados de convergencia en función de h o p, debido a que 
la solución analítica obtenida acarreará estos errores.
\begin{figure}[H]
    \centering
    \begin{subfigure}{0.32\textwidth}
        \includesvg[width=\linewidth]{ej3_p4_h80_sol.svg}
    \end{subfigure}
    \hfill
    \begin{subfigure}{0.32\textwidth}
        \includesvg[width=\linewidth]{ej3_p4_h80_ders.svg}
    \end{subfigure}
    \hfill
    \begin{subfigure}{0.32\textwidth}
        \includesvg[width=\linewidth]{ej3_p4_h80_curve.svg}
    \end{subfigure}
    \caption{Comparación de resultados.}
    \label{fig:tresgraficas}
\end{figure}
Resultados numéricos concretos:\\
Error Norma L2 = 0.1386\\
Error Norma H1 = 161.68\\
Error Norma Infinito = 2.0702\\
Estos resultados poseen errores heredados del calculo de la longitud y de la reparametrización.

\end{document}

